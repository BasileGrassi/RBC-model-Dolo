\documentclass[12pt]{article}
\usepackage[utf8]{inputenc}
%\usepackage[latin1]{inputenc}
\usepackage[T1]{fontenc}
\usepackage[francais,english]{babel}
\usepackage{graphicx}
\usepackage{amsmath}
\usepackage{amsfonts}
\usepackage{amssymb}
\usepackage{fancyhdr}
%\usepackage[nottoc, notlof, notlot]{tocbibind}
\usepackage{geometry}
%\geometry{ hmargin=3.2cm, vmargin=4cm }
%\usepackage{graphicx}
\usepackage{hyperref}

\DeclareMathOperator*{\Max}{Max}
\DeclareMathOperator*{\Argmax}{Argmax}
\newcommand{\E}[1]{\mathbb{E}_{#1}}
\newcommand{\dpp}[2]{\frac{\partial #1}{\partial #2}}
\newcommand{\ddpp}[2]{\frac{d #1}{d #2}}
\newcommand{\ti}[1]{\widetilde{#1}}
\newcommand{\ub}[2]{\underbrace{#1}_{#2}}
\newcommand{\ubt}[2]{\underbrace{#1}_{\text{#2}}}
\newcommand{\oomega}[0]{\overline{\omega}}
\newtheorem{lemma}{Lemma}
\newtheorem{definition}{Definition}
\newtheorem{proposition}{Proposition}
\newtheorem{assumption}{Assumption}

\setlength{\parindent}{20pt}
\setlength{\parskip}{1ex plus 4.ex minus .2ex}
%\linespread{1.2}



\title{Help File for Dolo}

\author{Basile Grassi\footnote{Paris School of Economics and Universit\'e Paris I Panth\'eon-Sorbonne}~ and Pablo Winant\footnote{Paris School of Economics}}

\date{\today}

\begin{document}
\maketitle

\section{The \texttt{YAML} file: describe the model}

\section{The matlab file}
Assuming that your model'matlab file is called \texttt{baseline-model.m}.

By runing the comande:

\texttt{>> model=baseline-model.m}

you will get a structure which contains all the information about your model describe in your yaml file. The model is given as:
\begin{align*}
s_{+1} &= G(s,x,\varepsilon) \tag{Transition}\label{Transition}\\
\E{t}&[F(s,x,\varepsilon,s_{+1},x_{+1}]=0 \tag{Arbitrage}\label{Arbitrage}
\end{align*}

This structure is organize as follow:
\begin{itemize}
	\item \texttt{model.s\_ss}: Steaty-state value of state variables.
	\item \texttt{model.x\_ss}: Steaty-state value of control variables.
	\item \texttt{model.params}: Value of parameters in a array. The order is the same as in the declaration of parameters in the asociated yaml file.
	\item \texttt{model.X}: Perturbation method decision rule. One can find the value of the control variable around the steady-state using the following formula:
	\begin{equation*}
	x_t=x_{ss}+ X(s_t-s_{ss})
	\end{equation*}
	\item \texttt{model.f}: Matlab function which give you the function $F$ used in (\ref{Arbitrage}). The function could be call with the following syntax:
	
	 \begin{center} \texttt{f(s,x,s$_{+1}$,x$_{+1}$,p)} where \texttt{p} stend for the parameters of the model.\end{center}
	 
	 One will get the value of the function and the value of the derivative. The output is given as 
	 
	 \begin{center} \texttt{[F,F$_s$,F$_x$,F$_{s_{+1}}$,F$_{x_{+1}}$]=f(s,x,s$_{+1}$,x$_{+1}$,p)}\end{center} 
	 where \texttt{F$_s$,F$_x$,F$_{s_{+1}}$,F$_{x_{+1}}$} are the derivatives with respect to the states, the controls, the next period states and the next period controls respectively.
	
	\item \texttt{model.g}: Matlab function which give you the function $G$ used in (\ref{Transition}). The function could be call with the following syntax:
	
		 \begin{center} \texttt{g(s,x,e,p)} where \texttt{p} stend for the parameters of the model.\end{center}
	
		 One will get the value of the function and the value of the derivative. The output is given as 
	 
	 \begin{center} \texttt{[G,G$_s$,G$_x$]=g(s,x,p)}\end{center}
	 where \texttt{G$_s$,G$_x$} are the derivatives with respect to the states and the controls respectively.
	
	
	\end{itemize}

\end{document}
